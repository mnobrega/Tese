% %%%%%%%%%%%%%%%%%%%%%%%%%%%%%%%%%%%%%%%%%%%%%%%%%%%%%%%%%%%%%%%%%%%%%%
% State of the art
% %%%%%%%%%%%%%%%%%%%%%%%%%%%%%%%%%%%%%%%%%%%%%%%%%%%%%%%%%%%%%%%%%%%%%%

\fancychapter{Estado da Arte}
\label{chap:ea}
Pequena introdu��o.

\section{Sec��o 1}
\label{sec:ea:sec1}
...

%exemplo de sec��o fantasma
\phantomsection
\label{phsec:ea:14443}
A tecnologia dos cart�es \textit{contactless} apoia-se na norma \textbf{\acs{ISO}/\acs{IEC} 14443}, que especifica a comunica��o com \textit{smart cards contactless}, usando sinais de r�dio a 13.56 MHz e com alcance t�pico de 10 cm. Nesta norma est�o especificados dois tipos de cart�es \textit{contactless} --- Tipo `A' e Tipo `B' --- que se diferenciam entre si nos m�todos de modula��o, esquemas de codifica��o e protocolos de inicializa��o usados. A norma divide-se ainda em quatro partes:

\begin{description}
	\item[Parte 1 (ISO/IEC 14443-1)] Physical characteristics
	\item[Parte 2 (ISO/IEC 14443-2)] Radio frequency power and signal interface
	\item[Parte 3 (ISO/IEC 14443-3)] Initialization and anticollision
	\item[Parte 4 (ISO/IEC 14443-4)] Transmission protocol
\end{description}

% Ensure that the next chapter starts in a odd page
\cleardoublepage