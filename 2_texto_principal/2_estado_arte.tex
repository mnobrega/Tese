% %%%%%%%%%%%%%%%%%%%%%%%%%%%%%%%%%%%%%%%%%%%%%%%%%%%%%%%%%%%%%%%%%%%%%%
% State of the art
% %%%%%%%%%%%%%%%%%%%%%%%%%%%%%%%%%%%%%%%%%%%%%%%%%%%%%%%%%%%%%%%%%%%%%%

\fancychapter{Estado da Arte}
\label{chap:ea}
Pequena introdu��o.

\section{Literatura Relacionada}
\label{sec:ea:sec1}
A gera��o actual de casas inteligentes tem tido uma maior evolu��o na intelig�ncia artificial do sistema central, em detrimento dos sistemas de monitoriza��o e controlo. A casa inteligente actual consiste em v�rios electrodom�sticos e outros dispositivos, com sensores, actuadores e/ou monitores biom�dicos, usados pelos residentes numa base di�ria. Em alguns casos a casa � inteiramente monitorizada recorrendo a tecnologias �udio-visuais.

Foram identificadas tr�s tecnologias de monitoriza��o:
\begin{itemize}
\item �udio-Visual
\item Sensores \textit{Wearable}\footnote{Sensores muito pequenos que s�o usados por uma pessoa embutidos na sua roupa.}
\item Sensores para localiza��o e monitoriza��o de diversos electrodom�sticos numa casa
\end{itemize}

\subsection{Monitoriza��o �udio-Visual}
\label{sec:ea:sec1:sec1}

\section{IEEE 802.15.4 e ZigBee}
\label{sec:ea:sec2}
Tecnologia ZigBee 802.15.4 e protocolo de encaminhamento AODV; 

\section{Sensores Wireless}
\label{sec:ea:sec3}
Sensores ZigBee dispon�veis no mercado para o cumprimento dos objectivos;

\section{Hardware Dom�tico Existente}
\label{sec:ea:sec4}
Solu��es de hardware dom�tico existente 

\section{Algoritmos de Localiza��o}
\label{sec:ea:sec5}
Diversas op��es dispon�veis. Vantagens e desvantagens; Tabela comparativa;
Descri��o matem�tica do HORUS; O esquema que eu vou usar difere na medida em que o c�lculo � feito na base station e n�o no mobile node

% Ensure that the next chapter starts in a odd page
\cleardoublepage