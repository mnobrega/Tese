% %%%%%%%%%%%%%%%%%%%%%%%%%%%%%%%%%%%%%%%%%%%%%%%%%%%%%%%%%%%%%%%%%%%%%%
% State of the art
% %%%%%%%%%%%%%%%%%%%%%%%%%%%%%%%%%%%%%%%%%%%%%%%%%%%%%%%%%%%%%%%%%%%%%%

\fancychapter{Resultados}
\label{chap:r}
Pequena introdu��o.

\section{Pot�ncia Recebida}
\label{sec:r:sec1}
Histogramas das pot�ncias recebidas para situacao parada, em movimento e com obstaculos; Correla��o entre amostras

\section{Cria�ao dos RadioMaps e RadioMapClusters}
\label{sec:r:sec2}
Demonstra��o do caminho escolhido para construir os radiomaps e mobilidade utilizada

\section{Localiza��o}
\label{sec:r:sec3}
Analise dos erros de posicao; Analise do boost de performance por causa do uso de clusters; An�lise do efeito do centro de massa e do time avg;

\section{Throuput}
\label{sec:r:sec4}
Analise do throuput nos diversos casos de estudo
Analise de pacotes perdidos

\section{Escalabilidade}
\label{sec:r:sec5}
Analise do ponto em que e necessario adicionar mais uma baseStation
Analise do sistema com mais que uma base station

\section{Toler�ncia a Falhas de N�s}
\label{}
Verificar os efeitos na localiza��o da remo��o de n�s da rede.


% Ensure that the next chapter starts in a odd page
\cleardoublepage