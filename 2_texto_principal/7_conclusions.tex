% %%%%%%%%%%%%%%%%%%%%%%%%%%%%%%%%%%%%%%%%%%%%%%%%%%%%%%%%%%%%%%%%%%%%%%
% Conclusions
% %%%%%%%%%%%%%%%%%%%%%%%%%%%%%%%%%%%%%%%%%%%%%%%%%%%%%%%%%%%%%%%%%%%%%%

\fancychapter{Conclus�es e Trabalho Futuro}
\label{chap:7}

Com o intuito de obter uma simula��o o mais fidedigna poss�vel, foi necess�rio ao longo deste trabalho, encontrar solu��es que n�o se limitassem a simular um ou outro aspecto do problema, mas sim o conjunto completo, de funcionalidades do sistema proposto. 

Grande parte do problema ficou resolvido com o \acs{MiXiM}, no entanto foi necess�rio encontrar um protocolo de encaminhamento adequado e um sistema de localiza��o que conjuntamente permitissem criar um sistema completamente funcional de monitoriza��o.

Este trabalho permitiu assim desenvolver uma solu��o simulada de um sistema de monitoriza��o de pessoas idosas, sem deixar de referenciar o hardware necess�rio e adquirir um conhecimento alargado do protocolo de encaminhamento \acs{AODV} e do sistema de localiza��o HORUS.

Futuramente poder� ser melhorado o protocolo de localiza��o para que deixe de ser necess�rio existir uma fase offline, utilizando n�s fixos que saibam a sua pr�pria localiza��o e que com isso criem de forma autom�tica de peri�dica novos mapas r�dio sem a necessidade de interven��o.

Outra possibilidade interessante seria implementar no n� m�vel um conjunto de camadas paralelo com r�dio bluetooth ligando assim esse n� tamb�m a uma rede \acs{BSN} simulada, o que permitiria recriar no ambiente de trabalho eventos humanos que despoletassem mensagens, como aumento da temperatura ou batimento card�aco.

A melhoramento do modelo de obst�culos ou at� mesmo a utiliza��o de n�s reais que permitissem chegar aos par�metros do modelo tamb�m seria uma possibilidade  interessante de trabalho futuro.

% Ensure that the next chapter starts in a odd page
\cleardoublepage