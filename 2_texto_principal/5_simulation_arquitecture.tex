% %%%%%%%%%%%%%%%%%%%%%%%%%%%%%%%%%%%%%%%%%%%%%%%%%%%%%%%%%%%%%%%%%%%%%%
% State of the art
% %%%%%%%%%%%%%%%%%%%%%%%%%%%%%%%%%%%%%%%%%%%%%%%%%%%%%%%%%%%%%%%%%%%%%%

\fancychapter{Arquitectura do Sistema}
\label{chap:im}
Pequena introdu��o.

\section{Pressupostos e Estrutura}
\label{sec:im:sec1}
Limita��es da framework que v�o diferir da realidade;
Explica��o de todos os intervenientes no sistema: n�s m�veis, est�ticos e de base;
A forma como est�o interligados; A forma como � feita a escalabilidade e distin��o entre redes de andares diferentes;
O tipo de n�s presentes no sistema.

\section{Ficheiros XML de Configura��o}
\label{sec:im:sec2}
RadioMap; RadioMapClusters; Normal standard; Esquema com os diversos ficheiros;

\section{Network Layer}
\label{sec:im:sec3}
Tipos de mensagens da camada Netw e fluxogramas como a forma como essas mensagens s�o tratadas por cada tipo de n�;
Estruturas que fazem parte da camada Netw utilizadas; Exemplo com imagens do AODV a funcionar;
NetwToApplicationInfo para transportar informa��o acerca da pot�ncia do sinal;

\subsection{\textit{Ad hoc On-Demand Vector Routing}}
\label{chap:4:sec:4.1}

\section{Application Layer}
\label{sec:im:sec4}
Explica��o da mensagem HoHuT e a forma como � usada para transportar informa��o;
Explica��o do comportamento, por fluxograma, de cada um dos app layers da camada App;

\subsection{HORUS}
\label{chap:5:sec:5.1}

% Ensure that the next chapter starts in a odd page
\cleardoublepage