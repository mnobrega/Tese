% %%%%%%%%%%%%%%%%%%%%%%%%%%%%%%%%%%%%%%%%%%%%%%%%%%%%%%%%%%%%%%%%%%%%%%
% State of the art
% %%%%%%%%%%%%%%%%%%%%%%%%%%%%%%%%%%%%%%%%%%%%%%%%%%%%%%%%%%%%%%%%%%%%%%

\fancychapter{Arquitectura do Sistema}
\label{chap:5}
%TODO
\textcolor{red}{TODO: Pequeno resumo do cap�tulo}

\section{Sistema de Monitoriza��o EMoS}
\label{chap:5:sec:1}
Prop�e-se nesta tese o \acf{EMoS}, uma solu��o simulada para o problema da monitoriza��o de pessoas em ambiente dom�stico. O \acs{EMoS} � uma rede \acs{WSN} constitu�da por diversos n�s wireless colocados de forma homog�nea numa casa.Embora o sistema possa efectuar a monitoriza��o de todo o tipo de pessoas, neste trabalho ser� focada a monitoriza��o de idosos ou pessoas com necessidades especiais.

O sistema � caracterizado pelos seguintes tipos de dispositivo:

\begin{itemize}
\item N� M�vel (\acf{MN})
\item N� Est�tico (\acf{SN}
\item Esta��o Base (\acf{BS})
\end{itemize}

\title{\textbf{N� M�vel}}

O n� m�vel � um sensor wireless 


\section{Ficheiros XML de Configura��o}
\label{sec:im:sec2}

\section{Network Layer}
\label{sec:im:sec3}

\subsection{\textit{Ad hoc On-Demand Vector Routing}}
\label{chap:4:sec:4.1}

\section{Application Layer}
\label{sec:im:sec4}

\subsection{HORUS}
\label{chap:5:sec:5.1}

% Ensure that the next chapter starts in a odd page
\cleardoublepage