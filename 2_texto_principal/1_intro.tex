% %%%%%%%%%%%%%%%%%%%%%%%%%%%%%%%%%%%%%%%%%%%%%%%%%%%%%%%%%%%%%%%%%%%%%%
% The Introduction:
% %%%%%%%%%%%%%%%%%%%%%%%%%%%%%%%%%%%%%%%%%%%%%%%%%%%%%%%%%%%%%%%%%%%%%%

\fancychapter{Introdu��o}
\label{chap:intro}
Um pequeno texto introdut�rio.

\section{Motiva��o}
\label{sec:intro:motivacao}
A raz�o do trabalho ou pesquisa desenvolvida.

\section{Objectivos}
\label{sec:intro:objectivos}
Objectivos.

\section{Principais Contribui��es}
\label{sec:intro:contribs}
Aquilo que o trabalho ou pesquisa atingiu.

\section{Organiza��o da Disserta��o}
\label{sec:intro:organizacao}
Esta disserta��o encontra-se organizada nos seguintes seis cap�tulos:
\begin{enumerate}
	\item \nameref{chap:intro}
	\item \nameref{chap:ea}
	\item \nameref{chap:trab1}
	\item \nameref{chap:trab2}
	\item \nameref{chap:trab3}
	\item \nameref{chap:conclusoes}
\end{enumerate}

O \autoref{chap:intro} inclui a introdu��o ao projecto, assim como os seus objectivos, contribui��es do trabalho desenvolvido e a presente explica��o da organiza��o da disserta��o.

O \autoref{chap:ea} ...

No \autoref{chap:trab1} ...

O \autoref{chap:trab2} ...

O \autoref{chap:trab3} ...

Finalmente, no \autoref{chap:conclusoes} s�o tiradas as conclus�es do trabalho efectuado, fazendo-se tamb�m refer�ncias ao trabalho futuro que pode ser feito sobre o apresentado nesta disserta��o.

\cleardoublepage