\begin{abstract}
The consistent increase in the world's elder population has been putting a lot of challenges regarding national development, sustainability of families and the ability of health care systems to provide for ageing populations. As wireless sensing technology continues to evolve, devices integrating low-power, low-bandwidth radios and a modest amount of storage, emerge due to considerable reduced costs. Wireless sensors based home monitoring systems provide a safe, sound and secure environment for elder people, enabling them to live in their own home as long as possible. This work introduces the \acf{EMoS}, a \acs{MiXiM} based framework, in which an \acf{AODV} protocol has been implemented together with a modified HORUS system, for tracking and monitoring, in a home enviroment, elder people or people with special needs. The results obtained from this research demonstrate the feasibility to build a monitoring system for elder care using a simulated environment in which several aspects of the hardware commercially available have been also discussed. 
\end{abstract}