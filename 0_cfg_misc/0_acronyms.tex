\begin{acronym}[CDA]
	\acro{AID}{\textit{Application Identifier}\acroextra{, identificador das aplica��es dos \ac{PICC}}}
	\acro{API}{\acroemph{Application Programming Interface}}
	\acro{ATM}{\acroemph{Automated Teller Machine}\acroextra{ (Caixa Autom�tica)}}
	\acro{CA}{\acroemph{Certification Authority}\acroextra{, Autoridade de Certifica��o}}
	\acro{CDA}{\acroemph{Combined Dynamic Data Authentication / Application Cryptogram Generation}}
	\acro{CRC}{\acroemph{Cyclic Redundancy Check}\acroextra{, um m�todo de detec��o de erros por \acroemph{hashing}.}}
	\acro{CT}{\acroemph{Countertop}\acroextra{, ou \acroemph{Counter Top}}}
	\acro{CVM}{\acroemph{Cardholder Verification Method}}
	\acro{DDA}{\acroemph{Dynamic Data Authentication}}
	\acro{DES}{\acroemph{Data Encryption Standard}\acroextra{, algoritmo de cifra de bloco que usa chaves sim�tricas.}}
	\acro{EMV}{As iniciais de Europay, MasterCard e VISA\acroextra{, as tr�s companhias que trabalharam originalmente no desenvolvimento da norma.}}
	\acro{GPRS}{\acroemph{General Packet Radio Service}}
	\acro{GSM}{\acroemph{Global System for Mobile Communications}}
	\acro{GUI}{\acroemph{Graphical User Interface}}
	\acro{ICC}{\acroemph{Integrated Circuit Card}}
	\acro{IDE}{\acroemph{Integrated Development Environment}\acroextra{ (Ambiente de Desenvolvimento Integrado)}}
	\acro{IEC}{\acroemph{International Electrotechnical Commission}\acroextra{ (Comiss�o de Electrotecnia Internacional), uma organiza��o que define normas relacionados com electricidade, electr�nica e tecnologias relacionadas.}}
	\acro{ISO}{\acroemph{International Organization for Standardization}\acroextra{ (Organiza��o Internacional de Normaliza��o), uma organiza��o que define normas industriais e comerciais.}}
	\acro{MAC}{\acroemph{Message Authentication Code}\acroextra{ (C�digo de Autentica��o de Mensagem), um m�todo de \acroemph{hashing} criptogr�fico.}}
	\acro{MB}{Multibanco\acroextra{, sistema de caixas autom�ticas usado em Portugal}}
	\acro{MB-PED}{Multibanco-\acs{PED}}
	\acro{MB-POS}{Multibanco-Point of Sale}
	\acro{MXI}{Mondex International\acroextra{, empresa que desenvolveu o Mondex Electronic Cash, um sistema de carteira electr�nica usado pela MasterCard.}}
	\acro{NFC}{\acroemph{Near Field Communication}}
	\acro{OS}{Operating System}
	\acro{OTA}{\acroemph{Over-the-air}}
	\acro{OTLIS}{Operadores de Transportes da Regi�o de Lisboa}
	\acro{PAN}{\acroemph{Primary Account Number}}
	\acro{PCD}{\acroemph{Proximity Coupling Device}}
	\acro{PDA}{\acroemph{Personal Digital Assistant}}
	\acro{PED}{\acs{PIN} \acroemph{Entry Device}}
	\acro{PICC}{\acroemph{Proximity Integrated Circuit Card}}
	\acro{PIN}{\acroemph{Personal Identification Number}}
	\acro{PK}{\acroemph{Public Key}\acroextra{, Chave P�blica}}
	\acro{PMB}{Porta-Moedas Multibanco}
	\acro{POS}{\acroemph{Point of Sale}\acroextra{ (Posto de Venda)}}
	\acro{RFID}{\acroemph{Radio Frequency Identification}}
	\acro{RSA}{As iniciais de Ron \textbf{Rivest}, Adi \textbf{Shamir}, e Leonard \textbf{Adleman}\acroextra{, que publicaram o algoritmo pela primeira vez. O RSA � um algoritmo de criptografia assim�trico.}}
	\acro{RTC}{\acroemph{Real Time Clock}}
	\acro{RTOS}{\acroemph{Real-Time Operating System}}
	\acro{SAM}{\acroemph{Secure Application Module}\acroextra{, tamb�m conhecido como \acroemph{Secure Access Module}}}
	\acro{SDA}{\acroemph{Static Data Authentication}}
	\acro{SDK}{\acroemph{Software Development Kit}}
	\acro{SIBS}{Sociedade Interbanc�ria de Servi�os\acroextra{, sociedade que gere, entre outros, a rede \acs{MB}.}}
	\acro{SO}{Sistema Operativo}
	\acro{STIB}{\acroemph{Soci�t� des Transports Intercommunaux de Bruxelles}}
	\acro{TLV}{\acroemph{Tag-Length-Value}\acroextra{ (Etiqueta-Comprimento-Valor), uma etiqueta conhecida, seguida do tamanho do campo e do valor respectivo}}
	\acro{TPA}{Terminal de Pagamento Autom�tico}
	\acro{XML}{\acroemph{eXtensible Markup Language}}
\end{acronym}