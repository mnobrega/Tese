\section{State Of The Art}

\subsection{Monitoring using Video or Audio}
Current smart home environments consist of several appliances and other devices, with sensors, actuators and/or biomedical monitors. These systems are used by the residents in a daily basis. In some situations the house is monitored using video and audio technologies, even tho they present some disadvantages like: high costs due to sophisticated equipments and specialized deployment, the need for a large bandwidth or privacy issues. 
Several state-of-the-art solutions were reviewed. In \cite{2} falls are detected. In order to reduce the number of false alarms, the system integrates a \acs{WSN} and a video system. Cameras activated by a wireless sensor tracking mechanism, are able to interpret the video signal and make decisions whether to call an emergency number or not. A voice communication \acs{IEEE} 802.15.4 is also discussed through the usage of state-of-the-art radios capable of transmitting voice.\\
In \cite{3} and \cite{4} an installed surveillance system is used to infer about the position of a resident. No interaction with the system is needed in order to locate the person. The usage of \textit{Smart Cameras} allows to resolve the privacy issue of data transmission through air, with the possibility of some kind of spoofing existing, which would presents serious security concerns to the user.\\
\cite{5} deploys another monitoring application in a care home. It refers to the need for more healthcare professionals and the small amount of time that each one of them has available for each elder. The volume of biomedical data gathered can improve the way that the care manager follows it's dependent.\\
\cite{6} refers to the term \textit{aging in place} which represent a movement where elders live in an independent and safe manner in their own homes. Monitoring of falls but also utilitarian functionalities are implemented such as object detection, calendar, video-conference and address book.\cite{7} uses video and audio to correctly deduce if a fall has happen.
%%%%%%%%%%%%%%%%%%%%%%%%%%%%%%%%%%%%%%%%%%%%%%%%%%%%%%%%%%%%%%%%%%%%%%%%%%%%%%%%%%%%%%%%%%%%%
%%%%%%%%%%%%%%%%%%%%%%%%%%%%%%%%%%%%%%%%%%%%%%%%%%%%%%%%%%%%%%%%%%%%%%%%%%%%%%%%%%%%%%%%%%%%%
%%%%%%%%%%%%%%%%%%%%%%%%%%%%%%%%%%%%%%%%%%%%%%%%%%%%%%%%%%%%%%%%%%%%%%%%%%%%%%%%%%%%%%%%%%%%%

\subsection{Monitoring using Wearable Sensors}
\label{ssec:part2}
The increasing reduction in size of wireless sensors is bringing to the market solutions that can track a person's health, independently of his location or activity. The possibility of smart clothes with built-in sensors sufficiently small and light to be carried without any discomfort enable the mass usage of such equipments in a medium term.\\
In \cite{8} the \acf{BSN} is addressed, as a solution to detect  

%%%%%%%%%%%%%%%%%%%%%%%%%%%%%%%%%%%%%%%%%%%%%%%%%%%%%%%%%%%%%%%%%%%%%%%%%%%%%%%%%%%%%%%%%%%%%
%%%%%%%%%%%%%%%%%%%%%%%%%%%%%%%%%%%%%%%%%%%%%%%%%%%%%%%%%%%%%%%%%%%%%%%%%%%%%%%%%%%%%%%%%%%%%
%%%%%%%%%%%%%%%%%%%%%%%%%%%%%%%%%%%%%%%%%%%%%%%%%%%%%%%%%%%%%%%%%%%%%%%%%%%%%%%%%%%%%%%%%%%%%

\subsection{Part Three}
\label{sec:part3}
Taking into account ...

\cite{1}