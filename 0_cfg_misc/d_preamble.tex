% Defines an additional alphabet... not required in most cases
% ------------------------------------------------------------
% \DeclareMathAlphabet{\mathpzc}{OT1}{pzc}{m}{it}

% PACKAGE fontenc:
% -----------------
% chooses T1-fonts and allows correct automatic hyphenation.
\usepackage[T1]{fontenc}
\usepackage[latin1]{inputenc}

% PACKAGE babel:
% ---------------
% The 'babel' package may correct some hyphenisation issues of latex. 
% However in most situations it is not required.
\usepackage[portuges]{babel}

% PACKAGE latexsym:
% -----------------
% Defines additional latex symbols. May be required for thesis with many math symbols.
%\usepackage{latexsym}

% PACKAGE amsmath, amsthm, amssymb, amsfonts:
% -------------------------------------------
% This package is typically required. Among many other things it adds the possibility
% to put symbols in bold by using \boldsymbol (not \mathbf); defines additional 
% fonts and symbols; adds the \eqref command for citing equations. I prefer the style
% "(x.xx)" for referering to an equation than to use "equation x.xx".
\usepackage{amsmath, amsthm, amssymb, amsfonts}

% PACKAGE multirow, colortbl, longtable:
% ---------------------------------------
% These packages are most usefull for advanced tables. The first allows to join rows 
% throuhg the command \multirow which works similarly with the command \multicolumn
% The second package allows to color the table (both foreground and background)
% The third package is only required when tables extend beyond the length of one page;
% which typically does not happen and should be avoided
\usepackage{multirow}
\usepackage{colortbl}
% \usepackage{longtable}

% PACKAGE graphics, epsfig, subfigure, caption:
% ---------------------------------------------
% Packages for figures... well you will certainly need these packages, with the exception
% of the 'caption' package. This only allows to define extra caption options.
% Notice that subfigure allows to place figures within figures with its own caption. It
% should be avoided to create an eps file with subfigures. That will mean that you won't be 
% able to reference those subfigures. Instead create an EPS file (the only graphics format supported
% by latex) for each of the subfigures and then use the command \subfigure (see below).
\usepackage{graphicx}
\usepackage{epsfig}
\usepackage[hang,small,bf]{subfigure}
% \usepackage[hang,small,bf]{caption}

% PACKAGE algorithmic, algorithm
% ------------------------------
% These packages are required if you need to describe an algorithm.
% \usepackage{algorithmic}
% \usepackage[chapter]{algorithm}

% PACKAGE natbib/cite
% -------------------
% The two packages are not compatible, and you should use one of the two. Notice however that the
% IEEE BiBTeX stylesheet is imcompatible with the natbib package. If using the IEEE format, use the 
% cite package instead
\usepackage[square,numbers,sort&compress]{natbib}
%\usepackage{cite}

% PACKAGE acronyum
% -----------------
% This package is most usefull for acronyms. The package garantees that all acronyms definitions are 
% given at the first usage. IMPORTANT: do not use acronyms in titles/captions; otherwise the definition 
% will appear on the table of contents.
\usepackage[printonlyused,withpage]{acronym}

% PACKAGE extra_functions
% -----------------
% My Personal package: defines the following commands:
% \fancychapter{chaptername) -> Prints a fancier chapter (you can also use the fancychapter package for this)
% \hline{width} -> use for a replacement of the \hline command
% \Mark1, \Mark2, \Mark3, ...
\usepackage{extra_functions}

% PACKAGE tocloft
% -----------------
% The tocloft package provides means of controlling the typographic design of the Table of Contents,
% List of Figures and List of Tables. New kinds of `List of . . . ' can be defined.
% The package has been tested with the tocbibind, minitoc, ccaption, subfigure, float, fncychap, and hyperref packages.
\usepackage[subfigure]{tocloft}

% PACKAGE babelbib
% -----------------
\usepackage[fixlanguage]{babelbib}
\selectbiblanguage{portuges}

% PACKAGE todonotes
% -----------------
% \usepackage{todonotes}

% PACKAGE IEEEtrantools
% -----------------
% Allows customization of the IEEEtran bibliography style
% \usepackage[retainorgcmds]{IEEEtrantools}

% PACKAGE fixltx2e
% -----------------
% Allows the \textsubscript{} command, among other fixes
\usepackage{fixltx2e}

% PACKAGE longtable
% -----------------
% Use this instead of tabular when you need footnotes inside a table
\usepackage{longtable}

% PACKAGE hyperref
% -----------------
% Set links for references and citations in document
% Some MiKTeX distributions have faulty PDF creators in which case this package will not work correctly
% Long live Linux :D
\usepackage{hyperref}
\hypersetup{ a4paper=true,
             colorlinks=false,
             citecolor=red,
             breaklinks=true,
%            bookmarks=true,		% commented because it generated a warning
             bookmarksnumbered=true,
             bookmarksopen=true,
             pdftitle={T�tulo do PDF},
             pdfauthor={Autor},
             pdfsubject={Tese de Mestrado},
             pdfcreator={LaTeX},
             pdfkeywords={}
}

\def\chapterautorefname{Cap�tulo}
\def\sectionautorefname{Sec��o}
\def\subsectionautorefname{Subsec��o}
\def\figureautorefname{Figura}
\def\tableautorefname{Tabela}

% PACKAGE hypcap
% -----------------
% In case you use the package hyperref to create a PDF, the links to tables or figures
% will point to the caption of the table or figure, which is always below the table or figure itself.
% Therefore the table or figure will not be visible, it is above the pointer and one has
% to scroll up in order to see it.
% If you want the link point to the top of the image you can use this package
\usepackage[all]{hypcap}

% PACKAGE breakurl
% -----------------
% Provides breakable URLs
% This package is designed to be loaded after hyperref to provide a breakable
% hyperlinked \url command under DVI output
%
% Note: apparently not needed when using pdfTeX
% \usepackage{breakurl}

% Set paragraph counter to alphanumeric mode
\renewcommand{\theparagraph}{\Alph{paragraph}~--}

% Page formatting... It was correct for my master thesis... not sure it is still correct
\hoffset 0in
\voffset 0in
\oddsidemargin 0.71cm
\evensidemargin 0.04cm
\marginparsep 0in
\topmargin -0.25cm
\textwidth 15cm
\textheight 23.5cm

\usepackage{fancyhdr}
\pagestyle{fancy}
\renewcommand{\chaptermark}[1]{\markboth{\thechapter.\ #1}{}}
\renewcommand{\sectionmark}[1]{\markright{\thesection\ #1}}
\fancyhf{} \fancyhead[LE]{\bfseries\nouppercase{\leftmark}}
\fancyhead[RO]{\bfseries\nouppercase{\rightmark}}
\fancyfoot[LE,RO]{\bfseries\thepage}
\renewcommand{\headrulewidth}{0.5pt}
\renewcommand{\footrulewidth}{0.5pt}
\addtolength{\headheight}{2pt} % make space for the rule
\fancypagestyle{plain}{%
   \fancyhead{} % get rid of headers
   \renewcommand{\headrulewidth}{0pt} % and the line
   \renewcommand{\footrulewidth}{0pt}
}
\fancypagestyle{blank}{%
   \fancyhf{} % get rid of headers and footers
   \renewcommand{\headrulewidth}{0pt} % and the line
   \renewcommand{\footrulewidth}{0pt}
}
\fancypagestyle{abstract}{%
   \fancyhead{}
   \renewcommand{\headrulewidth}{0pt}
   \renewcommand{\footrulewidth}{0.5pt}
}
\fancypagestyle{document}{%
	\fancyhf{} \fancyhead[LE]{\bfseries\nouppercase{\leftmark}}
	\fancyhead[RO]{\bfseries\nouppercase{\rightmark}}
	\fancyfoot[LE,RO]{\bfseries\thepage}
	\renewcommand{\headrulewidth}{0.5pt}
	\renewcommand{\footrulewidth}{0.5pt}
	\addtolength{\headheight}{2pt} % make space for the rule
}
\setcounter{secnumdepth} {5}
\setcounter{tocdepth} {5}
\renewcommand{\thesubsubsection}{\thesubsection.\Alph{subsubsection}}

\renewcommand{\subfigtopskip}{0.3 cm}
\renewcommand{\subfigbottomskip}{0.2 cm}
\renewcommand{\subfigcapskip}{0.3 cm}
\renewcommand{\subfigcapmargin}{0.2 cm}


% Custom commands for automatic text
% (remember to add {} after each one, or adjacent spaces will be suppressed)
\newcommand{\isocless}{\mbox{ISO/IEC 14443}}
\newcommand{\extlog}{\textit{Extended Logging}}
\newcommand{\emv}{\nameref{sec:ea:emv}}

\newcommand{\acroemph}[1]{\textit{#1}}