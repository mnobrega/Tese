
\section{Conclusion}
In order to get a simulation the closest to reality, it was necessary during the course of this thesis, to find solution that would me limited to simulating isolated aspects of the problem, but all the complete set of functionalities that would allow for a close to real simulation.

A big part of this problem was solved by finding MiXiM, but the lack of a good routing protocol and node tracking system, made it clear that it was necessary to find such a solution. 

This thesis allowed the development of a fully integrated system, called EMoS that would allow further work and resolve the issues that arose.

In the future a better solution for the tracking system can be found, removing the need for an offline process, which would be very time consuming in real conditions.

Another possibility is the implementation in the node of parallel stack of layers that would allow the same node to communicate with a bluetooh network, extending the simulation to a BSN network simulating biologic events like heartbeat or blood pressure.

The improvement of the obstacles model could also be achieved in order to get better simulation parameters.

% if have a single appendix:
%\appendix[Proof of the Zonklar Equations]
% or
%\appendix  % for no appendix heading
% do not use \section anymore after \appendix, only \section*
% is possibly needed

% use appendices with more than one appendix
% then use \section to start each appendix
% you must declare a \section before using any
% \subsection or using \label (\appendices by itself
% starts a section numbered zero.)
%

%\appendices
%\section{Proof of the First Zonklar Equation}
%Appendix one text goes here.

% you can choose not to have a title for an appendix
% if you want by leaving the argument blank
%\section{}
%Appendix two text goes here.