\vspace*{.2\textheight}
\begin{quote}
	\begin{center}
		\vspace*{.05\textheight}
		\textit{``Nunca � demasiado tarde para ser aquilo que sempre se quis ser.''} --- George Eliot
	\end{center}

\end{quote}

\clearpage

\thispagestyle{empty}
\cleardoublepage

\thispagestyle{plain}
\pdfbookmark{Agradecimentos}{Acknowledgments}
\begin{acknowledgments}
No culminar de uma longa etapa, s�o v�rias as pessoas que fa�o quest�o de mencionar neste trabalho, expressando os meus sinceros agradecimentos.

� minha fam�lia, em especial � minha mulher, Ana Bernardo pelo seu constante apoio e ao meu filho, Tom�s, por me ter dado, ainda dentro da barriga da m�e, a for�a e motiva��o para concluir t�o longa etapa.

Aos meus pais por terem durante tanto tempo dado o apoio e a motiva��o para nunca desistir nos momentos mais dif�ceis, por tudo o que me ensinaram e pela confian�a que sempre me transmitiram. 

Por ter aceite orientar a minha tese e por toda a ajuda proporcionada, ao meu Orientador, Professor Renato Nunes, pela disponibilidade demonstrada e pelas ideias e opini�es sugeridas que me proporcionaram reflex�o e determina��o para concluir este trabalho.

Ao meu Co-Orientador, Professor Ant�nio Grilo, por me ter dado a conhecer nas suas aulas o mundo das redes m�veis e um simulador t�o poderoso como o OMNeT++.

Ao meu patr�o e amigo, Pedro Dias, aos meus amigos Rodrigo Dias e Daniel Zacarias por me terem dado o tempo e a serenidade para terminar este trabalho.



\end{acknowledgments}