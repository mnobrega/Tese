\begin{resumo}
O aumento constante da popula��o idosa mundial tem criado uma enorme quantidade de desafios ao desenvolvimento nacional, � sustentabilidade das fam�lias e � capacidade dos sistemas de sa�de de darem suporte � popula��o idosa. � medida que a tecnologia dos sensores wireless evolui, dispositivos de baixo consumo, reduzida largura de banda e capacidade de armazenamento m�dio, surgem no mercado, com custos de aquisi��o bastante reduzidos. A monitoriza��o de ambientes dom�sticos baseada em sensores wireless, fornece um meio seguro e contido para pessoas idosas, permitindo que estas possam viver nas suas casas o m�ximo tempo poss�vel. Este trabalho introduz o \acf{EMoS}, um sistema desenvolvido no \acf{MiXiM},  onde foi implementado um protocolo de encaminhamento \acf{AODV} e um sistema de localiza��o baseado no HORUS, com a finalidade de monitorizar, num ambiente dom�stico, pessoas idosas ou com necessidades especiais. Os resultados obtidos desta investiga��o demonstram a viabilidade de construir um sistema simulado para monitoriza��o de pessoas num ambiente dom�stico, onde aspectos de hardware comercialmente dispon�vel foram tamb�m discutidos.
\end{resumo}

\begin{palavraschave}
Redes de Sensores, Pessoas Idosas, Protocolos de Encaminhamento, Localiza��o, MiXiM
\end{palavraschave}

\clearpage
\thispagestyle{empty}
\cleardoublepage

\begin{abstract}
The consistent increase in the world's elder population has been putting a lot of challenges regarding national development, sustainability of families and the ability of health care systems to provide for ageing populations. As wireless sensing technology continues to evolve, devices integrating low-power, low-bandwidth radios and a modest amount of storage, emerge due to considerable reduced costs. Wireless sensors based home monitoring systems provide a safe, sound and secure environment for elder people, enabling them to live in their own home as long as possible. This work introduces the \acf{EMoS}, a \acs{MiXiM} based framework, in which an \acf{AODV} protocol has been implemented together with a modified HORUS system, for tracking and monitoring, in a home environment, elder people or people with special needs. The results obtained from this research demonstrate the feasibility to build a monitoring system for elder care using a simulated environment in which several aspects of the hardware commercially available have been also discussed. 
\end{abstract}


\begin{keywords}
Sensor Networks, Elder Care, Routing Protocols, Indoor Location, MiXiM
\end{keywords}

\clearpage
\thispagestyle{empty}
\cleardoublepage